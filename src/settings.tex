% geometry package import, page size setting
\usepackage[a5paper, total={5in, 7.1in}]{geometry}

% lainputenc package, encoding setting
\usepackage[utf8]{luainputenc}

% LuaLaTex on-the-fly loader
\usepackage{luaotfload}

% fontspec package, UTF8 symbols support, requires cm-unicode package
\usepackage{fontspec}

% Use CMU fonts which have both latin and cyrillic letters
\setmainfont{CMU Serif}[Ligatures=TeX]
\setsansfont{CMU Sans Serif}[Ligatures=TeX]
\setmonofont{CMU Typewriter Text}

% Localization package instead of babel
% \usepackage{polyglossia}
% \setdefaultlanguage{czech}
% \setotherlanguage{english}
% \setotherlanguage{russian}

% babel package, multilang support
\usepackage[main=czech,english,russian]{babel}

% bookmark package, contents hyperlinks and bookmarks 
\usepackage{bookmark}

% Disable \ch macro from LuaLaTeX so that it can be overriden by songs package macro
\let\ch\undefined 

% songs package
\usepackage[chorded]{songs}

% graphicx package, easy insert of images etc.
%\usepackage{graphicx}
%\graphicspath{ {./images/} }

% graphicx package, easy insert of images etc.
%\notenamesin{A}{H}{C}{D}{E}{F}{G}

% afterpage package, allows easy insert of new pages where needed
% \usepackage{afterpage}

% Custom macro for a new blankpage \afterpage{\blankpage}
% \newcommand\blankpage{
    % \null
    % \thispagestyle{empty}
    % \addtocounter{page}{-1}
    % \newpage
% }

% Index setup
\newindex{mainsongsindex}{mainsongsindex}

% Contents font size setup
\renewcommand{\idxtitlefont}{\normalsize}

% Index numbering per page instead of song number
%\indexsongsas{mainsongsindex}{\thepage}

% Disable page numbering
\pagestyle{empty}

%--------------------------------------
% Refrain marked by "Ref" instead of a vertical line, see https://tex.stackexchange.com/questions/438894/how-to-change-the-position-of-the-song-number

% \usepackage{xpatch}

% \makeatletter

% \xpatchcmd{\justifyleft}
  % {\SB@cbarshift}
  % {\ifSB@inchorus\advance\leftskip\versenumwidth\fi}
  % {}{}

% \xpatchcmd{\justifycenter}
  % {\SB@cbarshift}{}
  % {}{}

% \xpatchcmd{\SB@par}
  % {\ifdim\cbarwidth>\z@\nobreak\else\SB@ilpenalty\fi}{}
  % {}{}

% \xpatchcmd{\brk}
  % {
    % \ifdim\cbarwidth=\z@%
    % \ifrepchorus\marks\SB@cmarkclass{}\fi%
    % \SB@breakpoint\brkpenalty%
  % }
  % {
    % \endgroup\egroup
    % \ifrepchorus\ifSB@gotchorus\else
      % \global\setbox\SB@chorusbox\vbox{
        % \unvbox\SB@chorusbox
        % \SB@chorusbar\SB@box
        % \unvcopy\SB@box
        % \SB@breakpoint\brkpenalty
      % }%
    % \fi
  % }
  % {}{}

% \xpatchcmd{\beginchorus}
  % {\vnumberedfalse}{\vnumberedtrue}
  % {}{}

% \xpatchcmd{\SB@@beginchorus}
  % {
    % \setbox\SB@box\vbox\bgroup\begingroup%
      % \ifchorded%
        % \def\SB@everypar{%
          % \vrule\@height\baselineskip\@width\z@\@depth\z@%
          % \gdef\SB@everypar{}%
        % }%
        % \everypar{\SB@everypar\everypar{}}%
      % \fi%
  % }
  % {
    % \setbox\SB@box\vbox\bgroup\begingroup
      % \ifvnumbered
        % \def\SB@everypar{%
          % \setbox\SB@box\hbox{Ref}%
           % \ifdim\wd\SB@box<\versenumwidth
             % \setbox\SB@box
             % \hbox to\versenumwidth{\unhbox\SB@box\hfil}%
           % \fi
          % \ifchorded\vrule\@height\baselineskip\@width\z@\@depth\z@\fi
          % \placeversenum\SB@box
          % \gdef\SB@everypar{}%
        % }
      % \else
        % \def\SB@everypar{
          % \ifchorded\vrule\@height\baselineskip\@width\z@\@depth\z@\fi
          % \gdef\SB@everypar{}
        % }
      % \fi
      % \everypar{\SB@everypar\everypar{}}
  % }
  % {}{}

% \xpatchcmd{\SB@endchorus}
  % {\SB@chorusbar\SB@box}{}
  % {}{}

%--------------------------------------
% Macro setup for single line display of verses without chords, see https://tex.stackexchange.com/questions/19813/creating-a-custom-songbook-with-the-songs-package

%The intended solution offered by the songs package is to use the \chordsoff and \chordson macros to turn chord-spacing on and off as needed to designate which verses/choruses are chorded and which are not. After \chordsoff, lines are single-spaced; and after \chordson, they are double-spaced.
%Note that if your songs are in a master file that gets loaded into multiple book types (both chorded and non-chorded), then you will probably want to create your own conditional macros for this similar to the following:
%
% Custom switch between diplaying chords everywhere
\newcounter{AlwaysDisplayChords}
\setcounter{AlwaysDisplayChords}{0}

\ifchorded
  \ifnum\value{AlwaysDisplayChords}=1
    \newcommand{\stopchords}{}
    \newcommand{\resumechords}{}
  \else
    \newcommand{\stopchords}{\chordsoff}
    \newcommand{\resumechords}{\chordson}
  \fi
\else
  \newcommand{\stopchords}{}
  \newcommand{\resumechords}{}
\fi

\newcommand{\stopchordsalways}{\chordsoff}
\newcommand{\resumechordsalways}{\chordson}
%
%Then you can use \stopchords and \resumechords instead of \chordsoff and \chordson to affect the chord book's line-spacing without affecting the lyric-only books.
%Although one can globally modify \baselineadj to force single-spacing of all lines not containing chords (e.g., by writing something like "\baselineadj=-14pt"), this has the potential disadvantage of single-spacing even lines within a chorded verse that happen to not contain a chord. This can be visually confusing to some musicians.
%---------------------------------------------

% Digits formatting macro to add leading zeroes, eg. 5 → 05, alternative to \two@digits
\newcommand\formatTwoDigits[1]{\ifnum#1<10 0\fi#1}

% Unnumbered songs
%\nosongnumbers

% Unnumbered verses
\noversenumbers

% Step between verses setup +- 2pt
\versesep=5pt plus 2pt minus 2pt

% Chord font setup
\renewcommand{\printchord}[1]{\rmfamily\bf#1}

% Setup of \textnote background
%\renewcommand{\notebgcolor}{white}

% Multiword chords without word splitting, better to use for a singular song only
%\MultiwordChords

% Number of document columns setup
\songcolumns{2}

% Global song numbering variable
\newcounter{song_numbering}

% obeylines fix, see https://tex.stackexchange.com/a/666610
\makeatletter
\renewcommand\SB@obeylines{%
  \let\obeyedline\SB@par%
  \obeylines%
  \let\@par\SB@@par%
}
\makeatother

%====================================================